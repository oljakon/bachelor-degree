\Introduction
В последние годы очень быстрыми темпами развивается область  обработки естественных языков. Во  многом  это связано с тем, что с каждым годом объём текстовой информации, используемой человечеством,  увеличивается,  и  растёт  потребность  в  более  эффективных алгоритмах  обработки  и  анализа  документов,  написанных  на  естественных языках. Особо важную роль играет возможность определить автора текста, основываясь на его стилистических признаках.

Выявление признаков авторского стиля позволяет установить принадлежность текста определенному человеку. Актуальность данного направления в компьютерной лингвистике обусловлена необходимостью в выявлении плагиата или в создании рекомендательной системы для нахождения похожжих текстов. Для определения стиля автора необходимо выделить характерные признаки из принадлежащих ему текстов.


Целью данной работы является разработка метода определения признаков авторского стиля для текстов на русском языке.

Для достижения этой цели ставятся следующие задачи:

\begin{enumerate}
	\item изучить предметную область;
    \item проанализировать существующие решения;
    \item проанализировать алгоритмы классификации;
    \item разработать метод определения признаков авторского стиля;
    \item рспроектировать структуру ПО для проведения исследования;
    \item реализовать ПО;
    \item провести апробацию предложенного метода.
\end{enumerate}

Разрабатываемый метод позволит учитывать классификацию текстов на русском языке по авторскому стилю на основании результатов морфологического анализа, не задействуя более трудоемкий и менее доступный синтаксический анализ.
